\documentclass[12pt,letterpaper]{article}
\usepackage[utf8]{inputenc}%acentos y caracteres en español
\usepackage[spanish]{babel}%Secciones y subsecciones en español teoremas ,etc.
\usepackage[total={18cm,21cm},top=2cm,left=2cm]{geometry}%configuración de página
\usepackage{amsmath,amssymb,amsfonts,latexsym}%símbolos matemáticos y texto normal
\usepackage{graphicx}%gráficos y figuras
\usepackage[xlines,table]{xcolor}%color del texto
\usepackage{verbatim}
\usepackage{hyperref}%enlaces
\usepackage{multicol}
\usepackage{dsfont}%Incierto OUO
\usepackage{booktabs}%aparte de \hline  adiciona \toprule y \bottomrule
\usepackage{tabularx}%tablas
\usepackage{rotating}
\usepackage{float}
\parskip=3pt%espacio entre lineas 
\parindent=3pt 
\definecolor{Nombre_del_color}{RGB}{245,254,254}%Color de fondo de la tabla

\title{Curso Autodidacto de tablas en LaTeX}
\author{Creado por: Carlos Raúl P.S.}
\begin{document}
\maketitle

\section{Preámbulo para la correcta creación de tablas}
Para poder crear tablas se necesita un preámbulos que contengan todos los aspectos que se van a utilizar en la tabla.
\begin{verbatim}
    \documentclass[12pt,letterpaper]{article}
    \usepackage[utf8]{inputenc}%acentos y caracteres en español
    \usepackage[spanish]{babel}%Secciones y subsecciones en español teoremas ,etc.
    \usepackage[total={18cm,21cm},top=2cm,left=2cm]{geometry}%config. de pág.
    \usepackage{amsmath,amssymb,amsfonts,latexsym}%símbolos matemáticos  
    \usepackage{graphicx}%gráficos y figuras
    \usepackage[xlines,table]{xcolor}%color del texto
    \usepackage{verbatim}
    \usepackage{hyperref}%enlaces
    \usepackage{multicol}
    \usepackage{dsfont}%incierto OUO
    \usepackage{booktabs}%aparte de \hline  adiciona \toprule y \bottomrule
    \usepackage{tabularx}%tablas
    \usepackage{rotating}
    \parskip=5pt%espacio entre lineas 
    \parindent=3pt 
    \definecolor{Nombre_del_color}{RGB}{245,254,254}%Color de fondo de la tabla
\end{verbatim}
Para saber la lista de códigos de colores que corresponde ver en \url{https://www.htmlcsscolor.com/hex/F5FEFE}
\\ 
Recurso bastante util generador de tablas en \url{https://www.tablesgenerator.com/}.
\subsection{Ejemplo de tabla}
\subsubsection{Tablas \textit{matemáticas}  }
Todo contenido en ellas estará en estilo matemático.
\[ 
    \begin{array}{l|c|c|r}
         \text{$x$ lcr indica la cantidad de columnas}&r=right=derecha&c=center&l=left=izquierda\\
         \text{La cantidad de filas } &\text{array}&\textit{matriz}&\textbf{arreglo}
     \end{array}
\]
\begin{verbatim}
    \[ 
    \begin{array}{l|c|c|r}
         \text{$x$ lcr indica la cantidad de columnas}&r=right=derecha
         &c=center&l=left=izquierda\\
         \text{La cantidad de filas } &\text{array}&\textit{matriz}&\textbf{arreglo}
     \end{array}
    \]
\end{verbatim}
La csmáre
\section{HTML5}
Por cuestiones de la veda estoy aprendiendo SVG y tiene mucho que ver con HTML 8u, a continuación se mostrará
un ejemplo de código HTM5 OpiO ...
\end{document}
