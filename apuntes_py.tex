\documentclass[12pt,letterpaper]{article}
\usepackage[spanish]{babel}
\usepackage[utf8]{inputenc}

\usepackage{booktabs}
\usepackage{tabularx}
\usepackage{float}
\usepackage{amsmath,amsfonts,amssymb}

\usepackage{xcolor}
\usepackage{listings}

\definecolor{codegreen}{rgb}{0,0.6,0}
\definecolor{codegray}{rgb}{0.5,0.5,0.5}
\definecolor{codepurple}{rgb}{0.58,0,0.82}
\definecolor{backcolour}{rgb}{0.95,0.95,0.92}

\lstdefinestyle{mystyle}{
    backgroundcolor=\color{backcolour},   
    commentstyle=\color{codegreen},
    keywordstyle=\color{cyan},
    numberstyle=\tiny\color{codegray},
    stringstyle=\color{codepurple},
    basicstyle=\ttfamily\footnotesize,
    breakatwhitespace=false,         
    breaklines=true,                 
    captionpos=b,                    
    keepspaces=true,                 
    numbers=left,                    
    numbersep=5pt,                  
    showspaces=false,                
    showstringspaces=false,
    showtabs=false,                  
    tabsize=2
}

\lstset{style=mystyle}



\author{CrlsRl}
\title{My notes of python autocourse }
\begin{document}
\maketitle
\section{Python basics}
I use by guide the sintaxis equivalence fortran functions.
\begin{table}[h!]
  \centering
  \begin{tabular}{r|l}
    For use only the keyboard remember& \texttt{Tab} and \texttt{Shift + Tab}\\
    Change tabs in VSCode & \texttt{alt+'number-of-the-tab'} \\
    Open terminal & \texttt{Wind + º}\\
    Enter in mode Mu o IDLE for phyton 3 & \texttt{python3 -v} \\
    Compile and execute file of python & \texttt{python3 namefile.py} \\

  \end{tabular}
  \caption{Descripción}
\end{table}
\subsection{Short ideas}
\begin{table}[h!]
  \centering
  \begin{tabular}{r}
    In python only exist two types of numbers variables the integers and\\ the floating-point numbers.\\
    \texttt{Win + Space} for change the keyboard distro.\\
    The \texttt{input()} in python acepted all type of variable do you write\\ 
    Exit fullscreen mode of VSCode with \texttt{F11}
  \end{tabular}
  \caption{Puntual notes}
\end{table}
\section{Flow control}
\begin{table}[h!]
  \centering
  \begin{tabular}{r|l}
    ... & ...
  \end{tabular}
  \caption{Descripción}
\end{table}
\subsection{Short ideas}
\begin{table}[h!]
  \centering
  \begin{tabular}{r}
    ... 
  \end{tabular}
  \caption{Puntual notes}
\end{table}
\begin{lstlisting}[language=Python, caption=\text{If, else, 'elif is equal else if' sintaxis of basics conditionals.} ]
  if variable1 == equivalence-condition1
    Order to do if is true the equivalence.
  elif variable2 ==  equivalence-condition2
    Order to do if is true the equivalence.
  elif variable3 == equivalence-condition3
    Order to do if is true the equivalence.
  more & more conditionals...  
#optional the sintaxis end with a final else
  else:
    Order to do if is false the previous equivalences.
\end{lstlisting}
\end{document}
