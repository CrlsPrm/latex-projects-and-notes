\documentclass[letterpaper,12pt]{article}
\usepackage{fancybox}
\usepackage{verbatim}
\title{Apuntes de Tipos de Cajas \LaTeX{Box}}
\author{Carlos Raul}
\date{27/05/2022}
\begin{document}
\maketitle
    \section{Apuntes}
    Las cajas se caracterizan por tres
elementos (longitudes):
\begin{verbatim}
    
    Altura sobre la línea base (\height), profundidad
    (\depth) (por ejemplo, la longitud del
    rabillo de la letra “p”) y anchura
    (\width). La suma de altura y profundidad
    se denomina \totalheight.
\end{verbatim} 
Ejemplo de preambulos
\begin{verbatim}
    \documentclass{article}
    \usepackage{amsmath}
    \usepackage[letterpaper, margin=1.5in, headheight=14pt]{geometry}
\end{verbatim}

    \subsection{Ejemplos}
    Caja box \framebox[3\width]{simple.}


\end{document}