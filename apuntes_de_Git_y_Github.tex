%%%%%%%%%%%%%%%%%%%%%%%%%%%%% Define Article %%%%%%%%%%%%%%%%%%%%%%%%%%%%%%%%%%
\documentclass{article}
\usepackage[utf8]{inputenc}
\usepackage[spanish]{babel}
%%%%%%%%%%%%%%%%%%%%%%%%%%%%%%%%%%%%%%%%%%%%%%%%%%%%%%%%%%%%%%%%%%%%%%%%%%%%%%%

%%%%%%%%%%%%%%%%%%%%%%%%%%%%% Using Packages %%%%%%%%%%%%%%%%%%%%%%%%%%%%%%%%%%
\usepackage{geometry}
\usepackage{graphicx}
\usepackage{amssymb, amsmath, amsthm}
\usepackage{empheq} 
\usepackage{mdframed}
\usepackage{booktabs}
\usepackage{lipsum}
\usepackage{psfrag}
\usepackage{pgfplots}
\usepackage{bm}
\usepackage{hyperref}
\usepackage{verbatim}
\usepackage{tabularx}%tablas
\usepackage{float}
\usepackage{color}
\usepackage{xcolor}


%%%%%%%%%%%%%%%%%%%%%%%%%%%%%%%%%%%%%%%%%%%%%%%%%%%%%%%%%%%%%%%%%%%%%%%%%%%%%%%

% Other Settings

%%%%%%%%%%%%%%%%%%%%%%%%%% Page Setting %%%%%%%%%%%%%%%%%%%%%%%%%%%%%%%%%%%%%%%
\geometry{a4paper}

%%%%%%%%%%%%%%%%%%%%%%%%%% Define an orangebox command %%%%%%%%%%%%%%%%%%%%%%%%
\renewcommand{\contentsname}{Contenido}
\renewcommand{\partname}{Parte}
\renewcommand{\appendixname}{Apéndice}
\renewcommand{\figurename}{Figura}
\renewcommand{\tablename}{Tabla}
\AtBeginDocument{\renewcommand\tablename{Tabla}}
\renewcommand{\abstractname}{Resumen}
\renewcommand{\refname}{Bibliografía}
\newcommand\orangebox[1]{\fcolorbox{ocre}{mygray}{\hspace{1em}#1\hspace{1em}}}
%%%%%%%%%%%%%%%%%%%%%%%%%%%%%%%%%%%%%%%%%%%%%%%%%%%%%%%%%%%%%%%%%%%%%%%%%%%%%%%

%%%%%%%%%%%%%%%%%%%%%%%%%%%% English Environments %%%%%%%%%%%%%%%%%%%%%%%%%%%%%
\newtheoremstyle{mytheoremstyle}{3pt}{3pt}{\normalfont}{0cm}{\rmfamily\bfseries}{}{1em}{{\color{black}\thmname{#1}~\thmnumber{#2}}\thmnote{\,--\,#3}}
\newtheoremstyle{myproblemstyle}{3pt}{3pt}{\normalfont}{0cm}{\rmfamily\bfseries}{}{1em}{{\color{black}\thmname{#1}~\thmnumber{#2}}\thmnote{\,--\,#3}}
\theoremstyle{mytheoremstyle}
\newmdtheoremenv[linewidth=1pt,backgroundcolor=shallowGreen,linecolor=deepGreen,leftmargin=0pt,innerleftmargin=20pt,innerrightmargin=20pt,]{theorem}{Theorem}[section]
\theoremstyle{mytheoremstyle}
\newmdtheoremenv[linewidth=1pt,backgroundcolor=shallowBlue,linecolor=deepBlue,leftmargin=0pt,innerleftmargin=20pt,innerrightmargin=20pt,]{definition}{Definition}[section]
\theoremstyle{myproblemstyle}
\newmdtheoremenv[linecolor=black,leftmargin=0pt,innerleftmargin=10pt,innerrightmargin=10pt,]{problem}{Problem}[section]
%%%%%%%%%%%%%%%%%%%%%%%%%%%%%%%%%%%%%%%%%%%%%%%%%%%%%%%%%%%%%%%%%%%%%%%%%%%%%%%

%%%%%%%%%%%%%%%%%%%%%%%%%%%%%%% Plotting Settings %%%%%%%%%%%%%%%%%%%%%%%%%%%%%
\usepgfplotslibrary{colorbrewer}
\pgfplotsset{width=8cm,compat=1.9}
%%%%%%%%%%%%%%%%%%%%%%%%%%%%%%%%%%%%%%%%%%%%%%%%%%%%%%%%%%%%%%%%%%%%%%%%%%%%%%%

%%%%%%%%%%%%%%%%%%%%%%%%%%%%%%% Title & Author %%%%%%%%%%%%%%%%%%%%%%%%%%%%%%%%
\title{\color{orange}\textbf{Git}}
\author{Carlos Raúl}
%%%%%%%%%%%%%%%%%%%%%%%%%%%%%%%%%%%%%%%%%%%%%%%%%%%%%%%%%%%%%%%%%%%%%%%%%%%%%%%

\begin{document}
    \maketitle
\section{Que es GIT?}
    Es un \textit{software} controlador de versiones de codigo abierto.
\subsection{Terminología}

    \begin{table}[h!]
        \centering
        \begin{tabular}{r|l}            
            Repositorio & Proyecto seguido en GIT\\
            Commit & Acción de actualizar el repositorio\\ 
            Ramas & Nuevos caminos del proyecto\\ 
            Master & Rama principal del proyecto\\ 
            Clon & Copia del repositorio\\
            Fork & Proyecto distinto a partir de otro\\ 
            BackUp & Copia del repositorio en caso de error de algún commit\\
            Fusiones Merge & Integración de una rama al Master\\
            Origin & Repositorio en la nube \textbf{Dev}
        
        \end{tabular}
        \caption{Terminología}
    \end{table}   
    \subsection{Como se trabaja con GIT}
    \subsubsection{Básico}
    Si se crea un proyecto desde cero se tipea en terminal \texttt{git init} o para clonar un repositorio \texttt{git clone}.
    \\ 
    El \texttt{Standing Area} o area de preparación es la interfaz intermedia antes de hacer el commit. 
    Cada \texttt{commit} debe ser específicamente descrito fin y utilidad.\\
    \begin{table}[h!]
        \centering
        \begin{tabular}{r|l}            
            \texttt{git init} o \texttt{git clone /url} & Inicio del proyecto o clonación de un proyecto anterior.\\
            \texttt{git add namefile.txt}& Agregar el archivo txt\\ 
            \texttt{git add *.txt} & Agrega todos los archivos con termino en .txt .\\
            \texttt{git status o git status -s}& Muestra el flujo sobre los archivos que se trabajan.\\ 
            \texttt{git add namefile.txt} &  Actualiza la  modificación de namefile.txt al Standing area.\\
            \texttt{git commit -m "Dscp."}&  Dscp.='Descripción de la modificación' así subes a  la rama.\\ 
            \texttt{git rm namefile.txt} & Elimina namefile.txt de tu espacio de trabajo.
            % \texttt{git restorer --staged namefile.txt}& Elimina nos quedamos en el minuto 39:27 mas o menos :P 
        \end{tabular}
        \caption{Flujo de trabajo}
    \end{table}   
    Así ciclicamente realizas las actualizaciones del proyecto. 
\section{Apéndice}
El \texttt{word wrap} sirve para la configuración de codigo así poder visualizar todo el codigo en una sola pestaña. \\ 
Aleluya...\\
    Creando archivos y directorios por terminal en linux... \\
\texttt{mkdir namedir} para crear carpetas desde terminal y escribimos \texttt{nvim} nos abre una nueva interfaz ahí digitamos el contenido  con  \texttt{i} luego salimos con \texttt{cntrl + c + :w namefile.txt}  
\section{Curiosidades y soluciones}
\subsection{Git no te permite subir carpetas vacías al repositorio}
\subsection{Si se tiene el problema de \texttt{! [rejected] master -> master (fetch first)} }
Problema al eliminar un archivo y crear una carpeta.
\begin{verbatim}
    As it is stated in the Error message you have to "fetch first".
    This worked for me. Use the command:
        git fetch origin master
    Then follow these steps to merge:
        git pull origin master
        git add .
        git commit -m 'your commit message'
        git push origin master
\end{verbatim}

\end{document}