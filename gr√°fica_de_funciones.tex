\documentclass[12pt,letterpaper]{article}
\usepackage[spanish]{babel}
\usepackage[utf8]{inputenc}
\usepackage{graphicx}
\usepackage{blindtext}
\usepackage{pgfplots}
\usepackage{amsmath}
\usepackage{amssymb}
\usepackage{epsfig}
\usepackage{color}
\usepackage{xcolor}
\usepackage{tikz}
\usepackage{booktabs}
\usepackage{xparse}
\usepackage{float}
\pgfplotsset{compat = newest}
\title{Grafica de funciones y curvas}
\author{Carlos Raúl P. S.}
\begin{document}
\pgfplotstableread{tabla.dat}{\table}
\maketitle
\date

\section{Grafica de funciones}
La siguiente $ \arctan(\varPhi(x)) $, $\forall x \in \mathbb{R}$ tabla muestra la grafica de funciones \textbf{Ojo: Buscar por el nombre del paquete y su manual así poder entender por completo}.\\\ 
Como Podemos apreciar, la grafica de las funciones es una curva continua.
\\ 
\begin{align*}
\frac{d}{dx}f(x) &= \frac{d}{dx}g(x)\\
\end{align*}
Se aprecia una interesante grafica de la funcion f(x) en el intervalo $[-1,1]$.
\begin{tikzpicture}
\begin{axis}[
    xmin = 0, xmax = 30,
    ymin = -1.5, ymax = 2.0,
    xtick distance = 2.5,
    ytick distance = 0.5,
    grid = both,
    minor tick num = 1,
    major grid style = {lightgray},
    minor grid style = {lightgray!25},
    width = \textwidth,
    height = 0.5\textwidth,
    xlabel = {$x$},
    ylabel = {$y$},]
 
% Plot a function
\addplot[
    domain = 0:30,
    samples = 200,
    smooth,
    thick,
    blue,
] {exp(-x/10)*( cos(deg(x)) + sin(deg(x))/10 )};
% Plot data from a file
\addplot[
    smooth,
    thin,
    red,
    dashed
] file[skip first] {coord.dat};
 
\end{axis}
 
\end{tikzpicture}
Como apreciamos en la tabla, la grafica de funciones es una curvas de funciones.
\subsubsection{Grafica tipo cuadro}
\begin{figure}[ht]
    
    \begin{center}
    \begin{tikzpicture}
            \centering
    \begin{axis}[
        title={Periodo $(s)$ vs. $masa_{as}$},
        xlabel={Temperature [\textcelsius]},
        ylabel={Solubility [g per 100 g water]},
        xmin=0, xmax=100,
        ymin=0, ymax=120,
        xtick={0,20,40,60,80,100},
        ytick={0,20,40,60,80,100,120},
        legend pos=north west,
        ymajorgrids=true,
        grid style=dashed,
        ]
        
        \addplot[
            color=blue,
        mark=square,
        ]
        coordinates {
            (0,23.1)(10,27.5)(20,32)(30,37.8)(40,44.6)(60,61.8)(80,83.8)(100,114)
            };
            \addplot[
                color=red,
                only marks, mark = x, mark size = 3pt,
                ]
            coordinates {
                (0,10)(10,20)(20,30)(30,35)(40,42)(60,60)(80,80)(100,100)
                };
                \addplot[
                    domain = 10:50,
                    samples = 200,
                    smooth,
                    thick,
                    cyan,
                    ] {1.5*x+10};
                    \legend{
                        CuSO\(_4\cdot\)5H\(_2\)O,
                        Aea,
                        Sape lokita 
                        }
                        
                    \end{axis}
            \end{tikzpicture}
        \end{center}
        \caption{Grafica de funciones tipo cuadro}
        \label{fig2}

    \end{figure}
\subsection{Ideal}
\blindtext
\begin{figure}[ht!]
    \begin{tikzpicture}
        \begin{axis}[
            xmin = 0, xmax = 10,
            ymin = 0, ymax = 1,
            xtick distance = 1,
            ytick distance = 0.10,
            grid = both,
            minor tick num = 1,
            major grid style = {lightgray},
            minor grid style = {lightgray!25},
        width = \textwidth,
        height = 0.75\textwidth,
        legend cell align = {left},
        legend pos = north west
        ]
     
        \addplot[blue, mark = *] table [x = {x}, y = {y1}] {\table};
        
        \addplot[red, only marks] table [x ={x}, y = {y2}] {\table};
        
        \addplot[teal, only marks, mark = x, mark size = 3pt] table [x = {x}, y = {y3}] {\table};
        
        \legend{
            Plot with marks and line, 
            Plot only with marks,
            Plot with other type of marks
            }
            
        \end{axis}
        
    \end{tikzpicture}
    \caption{Grafica de funciones tipo cuadro}
    \label{fig:cuadroideal}
    \end{figure}

    \blindtext 

\subsection{Por si las pulgas}
\blindtext
\begin{figure}[ht!]
    \begin{tikzpicture}
        
        \begin{axis}[
        width=10cm,
       height=4cm,
       scale only axis,
       xmin=0.8, xmax=3.2,
       xtick={1,2,3},
       xticklabels={Step1,Step2,Step3},
       xmajorgrids,
       ymin=0.7, ymax=1.1,
       ylabel={$\xi$},
       ymajorgrids,
       title={Stroke Ratio Comparison},
       axis lines*=left,
       %  line width=1.0pt,
       %  mark size=2.0pt,
       legend style ={ at={(1.03,1)}, 
       anchor=north west, draw=black, 
       fill=white,align=left},
       cycle list name=black white,
       smooth
    ]
    
    \addplot coordinates{
            (1,0.941517254116162)
            (2,0.833049791172753)
            (3,0.911012408209885) 
            };
            \addlegendentry{$L/D$=2.5};
            
            \addplot coordinates{
                (1,0.848646925806495)
            (2,0.889383162622147)
            (3,0.901265846062356) 
            };
       \addlegendentry{$L/D$=3.5};
       
       \addplot coordinates{
           (1,0.773775422607358)
           (2,0.836291579743709)
           (3,0.91821563038864) 
           };
           \addlegendentry{$L/D$=4.0};
           
           \addplot coordinates{
               (1,0.846412925696005)
               (2,0.909313371999676)
               (3,0.916886900310392) 
        };
        \addlegendentry{$L/D$=4.5};
        
        \addplot coordinates{
            (1,0.884302757051131)
            (2,0.941642806394511)
            (3,0.995341242858434) 
            };
            \addlegendentry{$L/D$=5.0};
            
            \addplot coordinates{
                (1,0.789821315674376)
            (2,0.981738703732297)
            (3,1.04121306622012) 
        };
        \addlegendentry{$L/D$=6.0};
    
    \end{axis}
    \end{tikzpicture}
    \caption{Figura loca}
    \label{figloca}
\end{figure}
\blindtext
\subsection{Grafica simple}
\begin{figure}
    \begin{center}
        \begin{tikzpicture}[domain=0:4]
            \draw[very thin,color=gray] (-0.1,-1.1) grid (3.9,3.9);
            \draw[color=black] (-0.2,0) -- (4.2,0) node[right] {$x$};
            \draw[color=black] (0,-1.2) -- (0,4.2) node[above] {$f(x)$};
            \draw[color=red] plot (\x,\x) node[right] {$f(x) =x$};
            \draw[color=blue] plot (\x,{sin(\x r)}) node[right] {$f(x) = \sin x$};
            \draw[color=orange] plot (\x,{0.05*exp(\x)}) node[right] {$f(x) = \frac{1}{20} \mathrm e^x$};
            \end{tikzpicture}
    \end{center}
    \caption{Grafica simple}
    \label{fig:simple}
\end{figure}
\blindtext
\subsection{La biblio:P}ç
Acá va la biblio
\begin{thebibliography}{9}
    \bibitem{lamport94}
      Leslie Lamport,
      \textit{\LaTeX: a document preparation system},
      Addison Wesley, Massachusetts,
      2nd edition,
      1994.
    \bibitem{chatarra}
      El chatarra,
      \textit{\LaTeX: a document preparation system},
      Addison Wesley, Massachusetts,
      2nd edition,
      1994.
\end{thebibliography}
\end{document}